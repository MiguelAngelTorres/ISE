%%%%%%%%%%%%%%%%%%%%%%%%%%%%%%%%%%%%%%%%%
% Short Sectioned Assignment LaTeX Template Version 1.0 (5/5/12)
% This template has been downloaded from: http://www.LaTeXTemplates.com
% Original author:  Frits Wenneker (http://www.howtotex.com)
% License: CC BY-NC-SA 3.0 (http://creativecommons.org/licenses/by-nc-sa/3.0/)
%%%%%%%%%%%%%%%%%%%%%%%%%%%%%%%%%%%%%%%%%

%----------------------------------------------------------------------------------------
%	PACKAGES AND OTHER DOCUMENT CONFIGURATIONS
%----------------------------------------------------------------------------------------

\documentclass[paper=a4, fontsize=11pt]{scrartcl} % A4 paper and 11pt font size

% ---- Entrada y salida de texto -----

\usepackage[T1]{fontenc} % Use 8-bit encoding that has 256 glyphs
\usepackage[utf8]{inputenc}
\usepackage{fourier} % Use the Adobe Utopia font for the document - comment this line to return to the LaTeX default

% ---- Idioma --------

\usepackage[spanish, es-tabla]{babel} % Selecciona el español para palabras introducidas automáticamente, p.ej. "septiembre" en la fecha y especifica que se use la palabra Tabla en vez de Cuadro

% ---- Otros paquetes ----

\usepackage{url} % ,href} %para incluir URLs e hipervínculos dentro del texto (aunque hay que instalar href)
\usepackage{amsmath,amsfonts,amsthm} % Math packages
%\usepackage{graphics,graphicx, floatrow} %para incluir imágenes y notas en las imágenes
\usepackage{graphics,graphicx, float} %para incluir imágenes y colocarlas

% Para hacer tablas comlejas
%\usepackage{multirow}
%\usepackage{threeparttable}

%\usepackage{sectsty} % Allows customizing section commands
%\allsectionsfont{\centering \normalfont\scshape} % Make all sections centered, the default font and small caps

\usepackage{fancyhdr} % Custom headers and footers
\pagestyle{fancyplain} % Makes all pages in the document conform to the custom headers and footers
\fancyhead[L]{Miguel Ángel Torres López}  % Header name
\fancyhead[C]{} % Empty center header
\fancyhead[R]{Ingeniería de Servidores - Prática 4} % Header title
\fancyfoot[L]{} % Empty left footer
\fancyfoot[C]{} % Empty center footer
\fancyfoot[R]{\thepage} % Page numbering for right footer
\renewcommand{\headrulewidth}{0pt} % Remove header underlines
\renewcommand{\footrulewidth}{0pt} % Remove footer underlines
\setlength{\headheight}{10pt} % Customize the height of the header

\numberwithin{equation}{section} % Number equations within sections (i.e. 1.1, 1.2, 2.1, 2.2 instead of 1, 2, 3, 4)
\numberwithin{figure}{section} % Number figures within sections (i.e. 1.1, 1.2, 2.1, 2.2 instead of 1, 2, 3, 4)
\numberwithin{table}{section} % Number tables within sections (i.e. 1.1, 1.2, 2.1, 2.2 instead of 1, 2, 3, 4)

\setlength\parindent{0pt} % Removes all indentation from paragraphs - comment this line for an assignment with lots of text

\newcommand{\horrule}[1]{\rule{\linewidth}{#1}} % Create horizontal rule command with 1 argument of height

\usepackage{listings}


%----------------------------------------------------------------------------------------
%	TÍTULO Y DATOS DEL ALUMNO
%----------------------------------------------------------------------------------------

\title{	
\normalfont \normalsize 
\textsc{\textbf{Ingeniería de Servidores (2016-2017)} \\ Doble grado en Ingeniería Informática y Matemáticas \\ Universidad de Granada} \\ [25pt] % Your university, school and/or department name(s)
\horrule{2pt} \\[0.4cm] % Thin top horizontal rule
\huge Memoria Práctica 4 \\ % The assignment title
\horrule{2pt} \\[0.5cm] % Thick bottom horizontal rule
}

\author{Miguel Ángel Torres López} % Nombre y apellidos

\date{\normalsize\today} % Incluye la fecha actual

%----------------------------------------------------------------------------------------
% DOCUMENTO
%----------------------------------------------------------------------------------------

\begin{document}

\maketitle % Muestra el Título

\newpage %inserta un salto de página

\tableofcontents % para generar el índice de contenidos

\newpage

\listoffigures

\listoftables

\newpage

%----------------------------------------------------------------------------------------
%	Cuestión 1
%----------------------------------------------------------------------------------------

\section{Seleccione, instale y ejecute un benchmark, comente los resultados.}

Para instalar Phoronix-test-suite se puede usar el gestor de paquetes apt, basta con ejecutar el comando sudo apt install phoronix-test-suite.
Una vez tengamos instalado el programa, se nos mostrará una mensaje de bienvenida con la versión del programa como en la 
Figura \ref{fig:figura1}. No obstante, no tenemos ningún benchmark descargado, luego para probarlo necesitaremos descargar alguno.
Con el comando \textit{phoronix-test-suite list-available-tests} se mostrarán en la terminal los tests disponibles para descargar. En mi
caso he elegido un test de GIMP, un editor de imagenes de código abierto. Al intentar hacer el test con el comando 
\textit{phoronix-suite-test benchmark system/gimp} empezará la descarga como se muestra en la Figura \ref{fig:figura3}.

\begin{figure}[H] 
\centering
\includegraphics[scale=0.3]{Figura1.png}  
\caption{Descarga de phoronix-test-suite.}\label{fig:figura1}
\end{figure}
\begin{figure}[H] 
\centering
\includegraphics[scale=0.3]{Figura3.png}  
\caption{Instalación del test de GIMP}\label{fig:figura3}
\end{figure}

\vspace{5mm}

Ahora al ejecutar el test nos preguntará que configuración queremos para el test. En mi caso he elegido la configuración 1. Tras varios
minutos de ejecución, nos salen los datos del test realizado. Como se puede observar en la Figura \ref{fig:figura4} mi test tiene una
salida de 19.10s. Ahora bien, no sabemos cuál es una puntuación buena ni cual es mala. Para comparar con otros máquinas podemos ir
a la página de OpenBenchmarking\cite{gimpbm} y comparar el resultado. La tabla sacada de la página puede consultarse en la 
Figura \ref{fig:figura5} y, a la vista de los resultados, la puntuación obtenida por mi máquina es media-baja. Es importante que a la hora
de realizar benchmark la máquina no esté estresada, es decir, no puede estar trabajando con otros procesos pesados pues los datos se
falsearían.

\begin{figure}[H] 
\centering
\includegraphics[scale=0.3]{Figura4.png}  
\caption{Ejecución del test.}\label{fig:figura4}
\end{figure}
\begin{figure}[H] 
\centering
\includegraphics[scale=0.3]{Figura3.png}  
\caption{Tabla de resultados del benchmark de GIMP.}\label{fig:figura5}
\end{figure}


%----------------------------------------------------------------------------------------
%	Cuestión 2
%----------------------------------------------------------------------------------------

\section{De los parámetros que le podemos pasar al comando ab ¿Qué significa -c 5? ¿Y -n 100? Monitorice la ejecución de ab contra alguna máquina ¿Cuantas tareas crea ab en el cliente?}

El \textit{benchmark} ab está especializado en comprobar cuantas peticiones es capaz de servir un servidor Apache.Como se especifica en el manual 
online\cite{manab}, el parámetro -c 5 establece la concurrencia del benchmark a 5, es decir, se realizarán 5 peticiones al servidor Apache por cada iteración
programada. Por otro lado, el parámetro -n 100 altera el número de peticiones y lo pone a 100. Esto quiere decir que se llevarán a cabo 100 peticiones
en la sesión de \textit{benchmark}. Estos dos parámetros son los básicos para administrar la prueba realizada contra Apache. Por ejemplo, en el caso expuesto, se
realizarían paquetes de 5 peticiones  hasta que las peticiones totales lleguen a 100. Por lo tanto se estarían realizando 20 iteraciones. Es interesante también usar el 
comando -k para mantener las conexiones realizadas vivas, de lo contrario la conexión se cerrará inmediatamente después de realizarse. Algo que no sucede de forma
habitual.

\vspace{5mm}

En la Figura \ref{fig:figura2} podemos observar la salida producia al realizar un \textit{benchmark} con ab y los parámetros especificados anteriormente.

\begin{figure}[H] 
\centering
\includegraphics[scale=0.3]{Figura2.png}  
\caption{Benchmark ab contra Apache en un servidor LAMP.}\label{fig:figura2}
\end{figure}

En principio podríamos pensar que \textit{ab} está usando 5 tareas ya que le hemos puesto que mande 5 peticiones concurrentemente. Podemos comprobar con el 
comando \textit{ps} \cite{manps} el número de tareas que tiene creadas \textit{ab} en un momento concreto. En la Figura \ref{fig:figura16} vemos que el sistema solo tiene una tarea asociada a \textit{ab}.

\begin{figure}[H] 
\centering
\includegraphics[scale=0.3]{Figura16.png}  
\caption{Número de tareas de ab comprobado con ps.}\label{fig:figura16}
\end{figure}

%------------------------------------------------


%----------------------------------------------------------------------------------------
%	Cuestión 3
%----------------------------------------------------------------------------------------

\section{Ejecute ab contra las tres máquinas virtuales una a una. ¿Cuál es la que proporciona mejores resultados? Muestre y coméntelos.}

Para hacer una comparación equitativa entre las tres máquinas es necesario cambiar el contenido de la página pues el contenidopor defecto de cada una de las 
versiones es diferente. En Windows Server el archivo iisstart está en el directorio \textit{intepub/wwroot}, en CentOS y Ubuntu Server se encuentra en el directorio
\textit{/var/www/html}. Para hacer esta prueba, escribiremos una página inicial simple como la que aparece en la Figura \ref{fig:figura6}. Es importante que el contenido
sea exactamente el mismo, pues \textit{ab} descargará el contenido de la web en cada petición.

\begin{figure}[H] 
\centering
\includegraphics[scale=0.3]{Figura6.png}  
\caption{Contenido de la página web.}\label{fig:figura6}
\end{figure}

Lo primero que vemos en esta comparación es que la dimensión del documento es la misma 69 bytes y la cantidad espacio ocupado por el envio del html es la 
misma, sin embargo, cada servidor transfiere una cantidad distinta de datos. Esto puede deberse a la forma en que cada sistema operativo trata el envío de 
información a través del protocolo tcp (datos redundantes, encriptación, información del envio, etc). La principal diferencia en esta característica la marca Windows
Server, que envía alrededor de 6000 bytes menos que sus dos competidores. Esto concuerda con el número de peticiones por segundo, Windows es el que más
peticiones realiza por segundo, seguido por Ubuntu de cerca y en último lugar y bastante alejado, CentOS.

\vspace{5mm}

Otro detalle a comentar es la velocidad transferencia. Aunque Windows tenga menos datos a enviar, Ubuntu tiene una mayor tasa de envío, es decir, envía más datos 
por segundo. Por lo tanto, Ubuntu gana ventaja y gestiona cada petición casi a la misma  velocidad que Windows. CentOS se queda alejado en la velocidad de servicio
de peticiones.

\vspace{5mm} 

A pesar de que en este caso Windows paracer mantener las prestaciones de Ubuntu, gran parte de esta ventaja viene dada por la baja cantidad de datos a transmitir.
En caso de grandes páginas con muchos datos que enviar, Ubuntu debería ser capaz de atender más peticiones por segundo. Los datos extraidos pueden verse en 
las siguientes figuras.

\begin{figure}[H] 
\centering
\includegraphics[scale=0.3]{Figura7.png}  
\caption{Ejecución de ab contra servidor en Ubuntu.}\label{fig:figura7}
\end{figure}
\begin{figure}[H] 
\centering
\includegraphics[scale=0.3]{Figura8.png}  
\caption{Ejecución de ab contra servidor en CentOS.}\label{fig:figura8}
\end{figure}
\begin{figure}[H] 
\centering
\includegraphics[scale=0.3]{Figura9.png}  
\caption{Ejecución de ab contra servidor en Windows}\label{fig:figura9}
\end{figure}


%------------------------------------------------


%----------------------------------------------------------------------------------------
%	Cuestión 4
%----------------------------------------------------------------------------------------

\section{Instale y siga el tutorial realizando capturas de pantalla y comentándolas. En vez de usar la web de JMeter, haga el experimento usando sus máquinas virtuales. ¿coincide con los resultados de ab?}

Al abrir JMeter nos aparece una ventana como la que se muestra en la Figura \ref{fig:figura10}. Como se muestra en esa misma figura, para crear hebras y poder enviar
peticiones de prueba al servidor tenemos que crear un grupo de hilos. 

\begin{figure}[H] 
\centering
\includegraphics[scale=0.3]{Figura10.png}  
\caption{Instalación de JMeter. Parte 1.}\label{fig:figura10}
\end{figure}

Ahora podemos configurar en la ventana del grupo de hilos la cantidad de peticiones, las hebras simultáneas, temporizadores y más complementos para sincronizar
las hebras. Seguiré la misma línea que en el ejercicio anterior para poder comparar bien los resultados. Luego indico que habrá 5 petciones simultaneas y 20 
iteraciones como se muestra en la Figura \ref{fig:figura11}.

\begin{figure}[H] 
\centering
\includegraphics[scale=0.3]{Figura11.png}  
\caption{Instalación de JMeter. Parte 2.}\label{fig:figura11}
\end{figure}

Ahora hay que añadir una acción para esas hebras. En nuestro caso tenemos que añadir una petición de HTTP por defecto. Hay que añadirla sobre las hebras tal y 
como se muestra en la Figura \ref{fig:figura12}.

\begin{figure}[H] 
\centering
\includegraphics[scale=0.3]{Figura12.png}  
\caption{Instalación de JMeter. Parte 3.}\label{fig:figura12}
\end{figure}

Configuramos en la ventana de petición HTTP solo la dirección, el resto será tomado con los valores por defecto, es decir, conexión con una página web. La dirección,
puesto que vamos a ejecutar contra una máquina virtual, es la dirección de la máquina en local.

\begin{figure}[H] 
\centering
\includegraphics[scale=0.3]{Figura13.png}  
\caption{Instalación de JMeter. Parte 4.}\label{fig:figura13}
\end{figure}

Por último y de la misma forma que añadimos la pestaña anterior, ahora añadimos un visor de resultados. Esto nos permitirá ver el resultado del test realizado contra la
máquina. El resultado es similar al de la Figura \ref{fig:figura14}. Para ejecutar el test y empezar a ver resultados solo hay que pinchar sobre el botón de lanzar (icono 
de iniciar) en la parte superior de la ventana mientras está abierta la pestaña de los hilos.

\begin{figure}[H] 
\centering
\includegraphics[scale=0.3]{Figura14.png}  
\caption{Instalación de JMeter. Parte 5.}\label{fig:figura14}
\end{figure}

Para añadir una petición concreta al test, adjuntamos una ventana de petición HTTP, elegiremos el argumento get y la ruta de la página inicial de nuestro servidor web.
Una vez ejecutado el test, empezaran a aparecer los datos en la gráfica. Podemos ver que el tiempo consumido en cada petición es similar al conseguido con \textit{ab} 
contra la máquina de Ubuntu Server. En el ejercicio anterior obteníamos que el 85\% de las peticiones se realizaban en menos de 30ms, en la gráfica de la 
Figura \ref{fig:figura17} vemos tambíen que los datos se acumulan por debajo de los 15ms. A medida que vamos realizando más peticiones, habría algunos casos 
particulares con un aumento de tiempo, que corresponde, más o menos, con lo ejecutado en \textit{ab}.

\begin{figure}[H] 
\centering
\includegraphics[scale=0.3]{Figura17.png}  
\caption{Resultados de la prueba.}\label{fig:figura17}
\end{figure}

%------------------------------------------------




%----------------------------------------------------------------------------------------
%	Cuestión 5
%----------------------------------------------------------------------------------------

\section{Programe un benchmark usando el lenguaje que desee o busque uno ya programado. Debe especificar:}
\begin{itemize}
  \item Objetivos del benchmark.
	\item Métricas.
	\item Instrucciones para su uso.
	\item Ejemplo de uso.
\end{itemize}

Vamos a detallar el uso del benchmark procBench, cuyo repositorio puede encontrarse en \cite{procbench}. El objetivo principal
de este benchmark es medir distintas características del rendimiento del procesador. Para ellos se realizan varios algoritmos
de generación de números de estructura matemática como números primos o la sucesión de fibonacci.El benchmark puede 
descargarse desde la página mencionada anteriormente. Puede obtenerse la versión compilada o el código fuente desde ésta.

\vspace{5mm}

El benchmark usa distintas unidades para medir la calidad del procesador. Por un lado mide la velocidad de ejecución de los 
distintos algoritmos mencionados anteriormente.
Además del tiempo, detalla algunas características de más bajo nivel, como son la velocidad de carga de los registros y de lectura
de memoria.

\vspace{5mm}

La ejecución de este benchmark es bastante sencilla. Si se ejecuta sin argumentos, se mostrarán varias opciones de ejeción.
Seleccionar los modos que se quieran ejecutar y volver a llamar al benchmark con los argumentos requeridos.
En mi caso, solo me interesa probar la velocidad de ejecución de la generación de números y la ver la velocidad de lectura de memoria,
luego especifico \textit{procbench -mp}. Podemos ver la salida en la Figura \ref{fig:figura15}. Podemos destacar que en la velocidad
de lectura de memoria cuanto más grande es el buffer de lectura más tarda en traer los datos. Pero además, vemos un claro escalón de 32KB
a 64KB, donde la velocidad se reduce más o menos a la mitad. Podemos intuir que existe algún impedimento hardware que
limita las lecturas a esa cantidad, por ejemplo los buses.



\begin{figure}[H] 
\centering
\includegraphics[scale=0.3]{Figura15.png}  
\caption{Ejecución de procbench -mp}\label{fig:figura15}
\end{figure}


\newpage

\bibliography{citando} %archivo citas.bib que contiene las entradas 
\bibliographystyle{plain} % hay varias formas de citar

\end{document}
