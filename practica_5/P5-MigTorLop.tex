%%%%%%%%%%%%%%%%%%%%%%%%%%%%%%%%%%%%%%%%%
% Short Sectioned Assignment LaTeX Template Version 1.0 (5/5/12)
% This template has been downloaded from: http://www.LaTeXTemplates.com
% Original author:  Frits Wenneker (http://www.howtotex.com)
% License: CC BY-NC-SA 3.0 (http://creativecommons.org/licenses/by-nc-sa/3.0/)
%%%%%%%%%%%%%%%%%%%%%%%%%%%%%%%%%%%%%%%%%

%----------------------------------------------------------------------------------------
%	PACKAGES AND OTHER DOCUMENT CONFIGURATIONS
%----------------------------------------------------------------------------------------

\documentclass[paper=a4, fontsize=11pt]{scrartcl} % A4 paper and 11pt font size

% ---- Entrada y salida de texto -----

\usepackage[T1]{fontenc} % Use 8-bit encoding that has 256 glyphs
\usepackage[utf8]{inputenc}
\usepackage{fourier} % Use the Adobe Utopia font for the document - comment this line to return to the LaTeX default

% ---- Idioma --------

\usepackage[spanish, es-tabla]{babel} % Selecciona el español para palabras introducidas automáticamente, p.ej. "septiembre" en la fecha y especifica que se use la palabra Tabla en vez de Cuadro

% ---- Otros paquetes ----

\usepackage{url} % ,href} %para incluir URLs e hipervínculos dentro del texto (aunque hay que instalar href)
\usepackage{amsmath,amsfonts,amsthm} % Math packages
%\usepackage{graphics,graphicx, floatrow} %para incluir imágenes y notas en las imágenes
\usepackage{graphics,graphicx, float} %para incluir imágenes y colocarlas

% Para hacer tablas comlejas
%\usepackage{multirow}
%\usepackage{threeparttable}

%\usepackage{sectsty} % Allows customizing section commands
%\allsectionsfont{\centering \normalfont\scshape} % Make all sections centered, the default font and small caps

\usepackage{fancyhdr} % Custom headers and footers
\pagestyle{fancyplain} % Makes all pages in the document conform to the custom headers and footers
\fancyhead[L]{Miguel Ángel Torres López}  % Header name
\fancyhead[C]{} % Empty center header
\fancyhead[R]{Ingeniería de Servidores - Prática 4} % Header title
\fancyfoot[L]{} % Empty left footer
\fancyfoot[C]{} % Empty center footer
\fancyfoot[R]{\thepage} % Page numbering for right footer
\renewcommand{\headrulewidth}{0pt} % Remove header underlines
\renewcommand{\footrulewidth}{0pt} % Remove footer underlines
\setlength{\headheight}{10pt} % Customize the height of the header

\numberwithin{equation}{section} % Number equations within sections (i.e. 1.1, 1.2, 2.1, 2.2 instead of 1, 2, 3, 4)
\numberwithin{figure}{section} % Number figures within sections (i.e. 1.1, 1.2, 2.1, 2.2 instead of 1, 2, 3, 4)
\numberwithin{table}{section} % Number tables within sections (i.e. 1.1, 1.2, 2.1, 2.2 instead of 1, 2, 3, 4)

\setlength\parindent{0pt} % Removes all indentation from paragraphs - comment this line for an assignment with lots of text

\newcommand{\horrule}[1]{\rule{\linewidth}{#1}} % Create horizontal rule command with 1 argument of height

\usepackage{listings}


%----------------------------------------------------------------------------------------
%	TÍTULO Y DATOS DEL ALUMNO
%----------------------------------------------------------------------------------------

\title{	
\normalfont \normalsize 
\textsc{\textbf{Ingeniería de Servidores (2016-2017)} \\ Doble grado en Ingeniería Informática y Matemáticas \\ Universidad de Granada} \\ [25pt] % Your university, school and/or department name(s)
\horrule{2pt} \\[0.4cm] % Thin top horizontal rule
\huge Memoria Práctica 5 \\ % The assignment title
\horrule{2pt} \\[0.5cm] % Thick bottom horizontal rule
}

\author{Miguel Ángel Torres López} % Nombre y apellidos

\date{\normalsize\today} % Incluye la fecha actual

%----------------------------------------------------------------------------------------
% DOCUMENTO
%----------------------------------------------------------------------------------------

\begin{document}

\maketitle % Muestra el Título

\newpage %inserta un salto de página

\tableofcontents % para generar el índice de contenidos

\newpage

\listoffigures

\listoftables

\newpage

%----------------------------------------------------------------------------------------
%	Cuestión 1
%----------------------------------------------------------------------------------------

\section{Al modificar los valores del kernel de este modo, no logramos que persistan después de reiniciar la máquina. ¿Qué archivo hay que editar para que los cambios sean permanentes?}

Tal y como se menciona en la documentación de RedHat\cite{sysctlconf} podemos usar el archivo \textit{sysctl.conf}. Este archivo perdurará entre 
reinicios y será leído por el kernel en cada inicio de la máquina, cargando así las preferencias establecidas en él. Por ejemplo, vamos a 
modificar un parámetro para que nuestra máquna no responda a un \textit{ping}. Por motivos de rapidez y no tener que reiniciar la máquina, voy a 
usar el comando \textit{sysctl}. Como puede verse en la Figura \ref{fig:figura1} los \textit{ping} dejan de contestarse.

\begin{figure}[H] 
\centering
\includegraphics[scale=0.3]{Figura1.png}  
\caption{Usando sysctl para cambiar características del sistema.}\label{fig:figura1}
\end{figure}

Nótese que al cambiar el archivo los cambios no tendrán efecto a no se que se obligue al kernel a cargar el archivo de nuevo. Esto se hace con 
opcion -p del comand sysctl\cite{mansysctl}.



\vspace{5mm}

%----------------------------------------------------------------------------------------
%	Cuestión 2
%----------------------------------------------------------------------------------------

\section{¿Con qué opcion se muestran todos los parámetros modificables en tiempo de ejecución?}

Como puede verse en el manual en línea\cite{mansysctl}, la opción -a lista todos los valores disponibles para modificar en tiempo de ejecución. 
Se puede usar el comando grep visto en prácticas anteriores para filtrar el resultado.

\subsection{Elija dos parámetros y explique, en dos líneas, qué función tienen.}

En la referencia \cite{modparam}, tenemos una gran cantidad de parámetros especificados con una pequeña descripción y su rango
de valores. Por ejemplo:

\begin{itemize}
	\item \textbf{threads-max}. Determina el número máximo de hebras que pueden crearse derivados de la utilidad para dividir 
tareas \textit{fork}. El valor mínimo permitido es 20, es decir, como mínimo se tiene que dejar derivarse 20 threads.
	\item \textbf{dmesg\_restrict}. Cuando está activado, valor 1, se obliga a los usuarios a tener privilegios para poder ejecutar
	la orden dmesg. Si está desactivado, valor  0, todos los usuarios pueden utilizar dmesg para ver las llamadas al kernel.
\end{itemize}

Ambos parámetros pueden verse con el comando \textit{sysctl -a | grep kernel} con sus valores predeterminados. En las Figuras
\ref{fig:figura2} y \ref{fig:figura3} pueden verse los de mi máquina.

\begin{figure}[H] 
\centering
\includegraphics[scale=0.4]{Figura2.png}  
\caption{Valor del parámetro threads-max}\label{fig:figura2}
\end{figure}
\begin{figure}[H] 
\centering
\includegraphics[scale=0.4]{Figura3.png}  
\caption{Valor del parámetro dmesg\_restrict}\label{fig:figura3}
\end{figure}
%------------------------------------------------

\vspace{5mm}

%----------------------------------------------------------------------------------------
%	Cuestión 3
%----------------------------------------------------------------------------------------

\section{Realice una copia de seguridad del registro, modifique el valor de una clave y restaure la copia de seguridad hecha. Compruebe que la clave tiene el valor original.}

En la documentación de Windows\cite{register} vienen algunas directrices para hacerlo. Puesto que lo hacemos con interfaz, el
cambio y la creación de claves es bastante intuitiva. El proceso a seguir para comprobar que la restauración del registro es
efectiva es el siguiente:

Primero abrimos el registro desde el buscador del inicio.

\begin{figure}[H] 
\centering
\includegraphics[scale=0.4]{Figura4.png}  
\caption{Abrir registro para edición.}\label{fig:figura4}
\end{figure}

Creamos una nueva clave de ejemplo y le damos un valor cualquiera.

\begin{figure}[H] 
\centering
\includegraphics[scale=0.4]{Figura5.png}  
\caption{Crear una nueva clave en el registro.}\label{fig:figura5}
\end{figure}

Desde el menú vemos las opciones de importar y exportar, la primera para cargar un registro y la segunda para exportar el 
registro actual a un archivo. Pinchamos en exportar y seleccionamos un lugar seguro donde guardar el archivo con la
configuración actual del registro.

\begin{figure}[H] 
\centering
\includegraphics[scale=0.4]{Figura6.png}  
\caption{Exportar registro a un lugar seguro.}\label{fig:figura6}
\end{figure}
\begin{figure}[H] 
\centering
\includegraphics[scale=0.4]{Figura7.png}  
\caption{Registro exportado a un lugar seguro.}\label{fig:figura7}
\end{figure}

Ahora cambiamos el valor de nuestra clave creada.

\begin{figure}[H] 
\centering
\includegraphics[scale=0.4]{Figura8.png}  
\caption{Cambio de valor de la clave creada.}\label{fig:figura8}
\end{figure}

En el mismo panel de antes, se nos da la opción de importar un registro. Desde el buscador que aparece elegimos el 
archivo que contiene los datos del registros anterior. Nos aparecerá un mensaje avisando de la sobreescritura de los valores de
las claves del registro.

\begin{figure}[H] 
\centering
\includegraphics[scale=0.4]{Figura9.png}  
\caption{Importar registro y recuperación del valor original de la clave.}\label{fig:figura9}
\end{figure}

%------------------------------------------------

\vspace{5mm}

%----------------------------------------------------------------------------------------
%	Cuestión 4
%----------------------------------------------------------------------------------------

\section{Enumere qué elementos se pueden configurar en Apache y en IIS para que Moodle funcione mejor.}

En la documentación\cite{moodle} de Moodle vienen varios parámetros recomendados para mejorar el rendimiento
de la plataforma. Entre estos hay una sección específica para Apache y otra para IIS.

\begin{itemize}
	\item \textbf{Apache Server}
			\begin{itemize}
				\item Establecer \textit{MaxRequestWorkers} como el 80\% de la memoria total divido por el máximo de memoria permitido por proceso de Apache. 
				\item Usar Apache 2 en lugar de Apache.
				\item Establecer \textit{KeepAliveTimeout} a 2 o 5 segundos. Para servidores muy saturados, considerar desactivar \textit{KeepAlive} solo si la página no contiene recursos dinámicos.
				\item Desactivar ExtendedStatus.
				\item Desactivar \textit{HostnameLookups} para reducir la latencia de DNS.
				\item Reducir el tiempo de \textit{TimeOut}.
				\item Considerar el uso de mod\_deflate para comprimir las respuestas del servidor.
			\end{itemize}
	\item \textbf{IIS}
			\begin{itemize}
				\item El equivalente a \textit{KeepAliveTimeout} es \textit{ListenBackLog}.
				\item Se puede cambiar \textit{MemCacheSize} para probar otros tamaños para la caché de IIS.
				\item Crear una DWORD llamada \textit{ObjectCacheTTL} para cambiar el tiempo que permanece un objeto en caché, por defecto a 30 segundos.
			\end{itemize}
\end{itemize}

%------------------------------------------------




\vspace{5mm}

%----------------------------------------------------------------------------------------
%	Cuestión 5
%----------------------------------------------------------------------------------------

\section{Ajuste la compresión en el servidor y analice su comportamiento usando varios valores para el tamaño del archivo a partir del cual comprimir. para comprobar que está comprimiendo puede usar el navegador o comandos como curl o lynx. Muestre capturas de pantalla de todo el proceso.}

En la documentación de Microsft\cite{compress} viene paso a paso explicado el proceso para activar y ajustar la compresión de IIS. Primero
buscamos el administrador de IIS en el buscador de inicio. Al abrirlo buscamos en la vista de características la opción de compresión como 
aparece en la Figura \ref{fig:figura10}

\begin{figure}[H] 
\centering
\includegraphics[scale=0.4]{Figura10.png}  
\caption{Administrador de IIS}\label{fig:figura10}
\end{figure}

Se nos abrirá un panel como el que aparece en la Figura \ref{fig:figura11}. Por ahora, vamos a hacer una prueba tal cual están los datos por 
defecto, así que dejaremos todas las casillas como se muestran en la Figura.

\begin{figure}[H] 
\centering
\includegraphics[scale=0.4]{Figura11.png}  
\caption{Valores por defecto de la compresión}\label{fig:figura11}
\end{figure}

Con motivo de comprobar si la página verdaderemente se comprime o no, modificaré su contenido para añadirle más peso. En la práctica dejé solo
un mensaje saludando, ahora adjuntaré dos fotografías y buen conjunto de mensajes como en la Figura \ref{fig:figura12}. Esto hará que nuestra
página se comprima con los valores por defectos de compresión.

\begin{figure}[H] 
\centering
\includegraphics[scale=0.4]{Figura12.png}  
\caption{Servidor web con contenido añadido.}\label{fig:figura12}
\end{figure}

Podemos ver en la Figura \ref{fig:figura13} como usando la orden curl\cite{mancurl} sacamos la información de las peticiones al servidor web. 
Observamos que en la cabecera se nos especifica que se ha usado gzip para comprimir los datos transmitidos.


\begin{figure}[H] 
\centering
\includegraphics[scale=0.4]{Figura13.png}  
\caption{Información sobre la petición al servidor web.}\label{fig:figura13}
\end{figure}

Ahora modificamos el valor mínimo que el archivo ha de tener para ser comprimido. Pondremos un número muy alto para asegurarnos de que nuestra
página no sobrepasa el mínimo. 

\begin{figure}[H] 
\centering
\includegraphics[scale=0.4]{Figura14.png}  
\caption{Cambiando el valor de las preferencias de compresión}\label{fig:figura14}
\end{figure}

En la Figura \ref{fig:figura15} tenemos el resultado de ejecutar de nuevo curl. Ahora el mínimo de peso para empezar a comprimir es más grande
que el peso de nuestra página y, por lo tanto, la compresión ya no es realizada al enviar los datos. Podemos ver también que el tamaño de los 
datos enviados ha crecido. Esto se debe a que anteriormente se estaba comprimiendo al enviar la información.

\begin{figure}[H] 
\centering
\includegraphics[scale=0.4]{Figura15.png}  
\caption{Información sobre la petición al servidor con mínimo para compresión alterado}\label{fig:figura15}
\end{figure}

%------------------------------------------------

\vspace{5mm}

%----------------------------------------------------------------------------------------
%	Cuestión 6
%----------------------------------------------------------------------------------------

\section{Elija un servicio y modifique un parámetro para mejorar su comportamiento. Monitorice el servicio antes y después de la modificación del parámetro aplicando cargas al sistema (antes y después) mostrando los resultados de la monitorización.}

Dado que hemos visto en un ejercicio anterior como mejorar el rendimiento de un servidor apache modificando varios parámetros, 
ahora vamos a ponerlo en práctica para comprobar si realmente se produce una mejora. Primero haremos una prueba con la 
herramienta \textit{ab}\cite{manab} vista en prácticas anteriores para ver el rendimiento base del servidor web.
 
\begin{figure}[H] 
\centering
\includegraphics[scale=0.4]{Figura18.png}  
\caption{Rendimiento base del servidor web.}\label{fig:figura18}
\end{figure}

Siguiendo las indicaciones de la cuestión 4, vamos a modificar el parámetro \textit{KeepAliveTimeout} y lo vamos a reducir. Para
ello vamos al archivo de configuración del servidor Apache, situado por defecto en \textit{/etc/apache2/apache2.conf}. Como se
puede ver en la figura \ref{fig:figura16} el archivo de configuración contiene una pequeña explicación del parámetro, una 
explicación más detallada puede verse en la documentación de Apache\cite{apacheargs}.  

\begin{figure}[H] 
\centering
\includegraphics[scale=0.4]{Figura16.png}  
\caption{Cambio de KeepAliveTimeout en el archivo de configuración.}\label{fig:figura16}
\end{figure}

Veamos ahora como ha mejorado el rendimiento con respecto a la anterior ejecución. Se ha reducido el tiempo de respuesta de cada
petición y se ha aumentado ligeramente la velocidad de transferencia de datos como puede verse en la Figura \ref{fig:figura19} 

\begin{figure}[H] 
\centering
\includegraphics[scale=0.4]{Figura19.png}  
\caption{Repercusión en el servidor web tras cambiar KeepAliveTimeout.}\label{fig:figura19}
\end{figure}

Ahora vamos a dar un paso más, vamos a desactivar el \textit{KeepAlive}. Para ello volvemos a modificar el archivo de 
configuración como puede verse en la Figura \ref{fig:figura17}.

\begin{figure}[H] 
\centering
\includegraphics[scale=0.4]{Figura17.png}  
\caption{Cambio de KeepAlive en el archivo de configuración.}\label{fig:figura17}
\end{figure}

El nuevo test de ab ejecutado contra el servidor web nos da la salida que aparece en la Figura \ref{fig:figura20}.Volvemos a 
ver una tímida mejoría en las características medidas. La mejora es bastante pequeña a causa del tamaño de la web,que hace que
los datos transferidos sean muy pocos y haga menos significativo el cambio de los parámetros. 

\begin{figure}[H] 
\centering
\includegraphics[scale=0.4]{Figura20.png}  
\caption{Repercusión en el servidor web tras desactivar KeepAlive.}\label{fig:figura20}
\end{figure}

%------------------------------------------------

\newpage

\bibliography{citando} %archivo citas.bib que contiene las entradas 
\bibliographystyle{plain} % hay varias formas de citar

\end{document}
