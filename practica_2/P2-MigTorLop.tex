\input{preambuloSimple.tex}

%----------------------------------------------------------------------------------------
%	TÍTULO Y DATOS DEL ALUMNO
%----------------------------------------------------------------------------------------

\title{	
\normalfont \normalsize 
\textsc{\textbf{Ingeniería de Servidores (2016-2017)} \\ Doble grado en Ingeniería Informática y Matemáticas \\ Universidad de Granada} \\ [25pt] % Your university, school and/or department name(s)
\horrule{2pt} \\[0.4cm] % Thin top horizontal rule
\huge Memoria Práctica 2 \\ % The assignment title
\horrule{2pt} \\[0.5cm] % Thick bottom horizontal rule
}

\author{Miguel Ángel Torres López} % Nombre y apellidos

\date{\normalsize\today} % Incluye la fecha actual

%----------------------------------------------------------------------------------------
% DOCUMENTO
%----------------------------------------------------------------------------------------

\begin{document}

\maketitle % Muestra el Título

\newpage %inserta un salto de página

\tableofcontents % para generar el índice de contenidos

\newpage

\listoffigures

\listoftables

\newpage

%----------------------------------------------------------------------------------------
%	Cuestión 1
%----------------------------------------------------------------------------------------

\section{Yum.}
\subsection{Liste los argumentos de yum necesarios para instalar, buscar y eliminar paquetes.}

Como se menciona en el manual en linea\cite{manyum} los argumenos necesarios para las tareas básicas son:
\begin{itemize}
	\item \textbf{Instalar} - install package
	\item \textbf{Buscar} - search string
	\item \textbf{Eliminar} - remove | erase package
\end{itemize} 

\vspace{6mm}

\subsection{¿Qué ha de hacer para que yum pueda tener acceso a Internet en el PC del aula?}

El problema del PC del aula es la situación dentro de un proxy. Para indicar a yum como pasarlo\cite{proxyyum} hay que especificarlo en el
archivo de configuración /etc/yum.conf con usuario con privilegios. Como estamos en CentOS, usaremos vi para añadir la siguiente
linea al archivo mencionado:

\textit{proxy = http://stargate.ugr.es:3128}

\vspace{6mm}

\subsection{¿Cómo añadimos un nuevo repositorio?}

En la documentación de RedHat\cite{redyum} se especifica como incorporar nuevos repositorios a yum. El comando \textbf{yum-config-manager} es muy
útil a la hora de administrar la configuración de yum. Añadiendo la terminacion \textbf{--add-repo dir-url} añadimos el repositorio dir-url.
Como una medida de seguridad, \textbf{yum-config-manager --(enable or disable) repositorio} permite o no permite descargar del repositorio. Para poder utilizar yum-config-manager es necesario instalar el paquete yum-utils.

En la Figura \ref{fig:figura12} se muestra como añadir el repositorio para descargar Skype. Cabe destacar que todas las acciones anteriores tienen que ser ejecutadas
en modo administrador.

\begin{figure}[H] 
\centering
\includegraphics[scale=0.4]{Figura12.png}  
\caption{Incluir un repositorio en yum. Realizada el 16 de Abril de 2017.} \label{fig:figura12}
\end{figure}
%------------------------------------------------





%----------------------------------------------------------------------------------------
%	Cuestión 2
%----------------------------------------------------------------------------------------

\section{Apt.}

%------------------------------------------------

\subsection{Liste los argumentos de pat necesarios para instalar, buscar y eliminar paquetes.}

Como se menciona en el manual en linea\cite{manapt} los argumenos necesarios para las tareas básicas son:
\begin{itemize}
	\item \textbf{Instalar} - apt-get install package
	\item \textbf{Buscar} - apt	-cache search string
	\item \textbf{Eliminar} - apt-get remove package
\end{itemize}

\vspace{6mm}

\subsection{¿Qué ha de hacer para que apt pueda tener acceso a Internet en el PC del aula?}

Según la documentación en \cite{proxyapt}, basta con añadir una linea al archivo /etc/apt/apt.conf para especificar que nos encontramos un proxy.
Con la orden \textit{Acquire::http::Proxy "http://stargate.ugr.es:3128";} podremos pasar el proxy con apt.
\vspace{6mm}

\subsection{¿Cómo añadimos un nuevo repositorio?}

En el manual\cite{repoapt} se menciona que se puede cambiar el archivo /etc/apt/source.list para añadir repositorios. Sin embargo,
se recomienda desde la versión 9.10 usar el comando add-apt-repository ppa:[url] para introducir nuevos repositorios.


%----------------------------------------------------------------------------------------
%	Cuestión 3
%----------------------------------------------------------------------------------------

\section{Puertos y Firewalls.}


\subsection{¿Con qué comando puede abrir/cerrar un puerto usando ufw? Muestre un ejemplo de cómo lo ha hecho.}

Con el comando ufw, que puede consultarse en el manual\cite{manufw}, se pueden establecer reglas para el firewall. Dos de las reglas que se pueden construir son
\textit{ufw allow [puerto]} y \textit{ufw deny [puerto]}. Estas dos reglas permiten respectivamente abrir y cerrar un puerto en la máquina. Para hacer efectivo el firewall y
así activar las reglas es necesario ejecutar el comando \textit{ufw enable}. El resultado de estos comandos puede verse en la Figura \ref{fig:figura13}.

\begin{figure}[H] 
\centering
\includegraphics[scale=0.4]{Figura13.png}  
\caption{Abrir y cerrar puertos con ufw. Realizada el 17 de Abril de 2017.} \label{fig:figura13}
\end{figure}

\vspace{6mm}

\subsection{¿Con qué comando puede abrir/cerrar un puerto usando firewall-cmd en CentOS? Muestre un ejemplo de cómo lo ha hecho.}

Para abrir y cerrar un puerto con firewall-cmd en centos basta con agregar dos argumentos a este comando. De forma similar al apartado anterior, --add-port=[port]/[protocol] 
añade una regla y --remove-port=[port]/[protocol] la quita. Pueden consultarse este y otros argumentos en la documentación oficial de RedHat\cite{redports}.

\begin{figure}[H] 
\centering
\includegraphics[scale=0.4]{Figura14.png}  
\caption{Abrir y cerrar puertos con firewall-cmd. Realizada el 17 de Abril de 2017.} \label{fig:figura14}
\end{figure}

\vspace{6mm}

\subsection{Utilice el comando nmap para ver que, efectivamente, los puertos estan accesibles.}

Otra forma de consultar el estado de los puertos es con el comando nmap. Puede verse un ejemplo en la Figura \ref{fig:figura15}.

\begin{figure}[H] 
\centering
\includegraphics[scale=0.4]{Figura15.png}  
\caption{Comprobar el estado de los puertos con nmap. Realizada el 17 de Abril de 2017.} \label{fig:figura15}
\end{figure}

%------------------------------------------------




%----------------------------------------------------------------------------------------
%	Cuestión 4 -
%----------------------------------------------------------------------------------------

\section{¿Qué diferencia hay entre telnet y ssh?}

Como se describe en \cite{telnetvsssh} telnet usa el puerto 23 mientras que ssh usa el puerto 22 por defecto. No obstante, ambos son protocolos de red usados
para acceder remotamente a dispositivos y manejarlos o administrarlos. La gran diferencia la encontramos en el modo de comunicación. Los mensajes en telnet 
se escriben en texto plano, lo que hace vulnerable la línea de información transportada. Ssh, considerado el sucesor de telnet, permite el cifrado de la información 
por medio de claves públicas y privadas, lo cual hace más seguro la comunicación con el dispositivo remoto. 

%------------------------------------------------




%----------------------------------------------------------------------------------------
%	Cuestión 5
%----------------------------------------------------------------------------------------

\section{¿Para qué sirve la opción -X? Ejecute remotamente, es decir, desde la máquina anfitriona (si tiene Linux) o desde la otra máquina virtual, el comando gedit en una sesión abierta con ssh. ¿Qué ocurre?}

Consultando el manual en linea\cite{manssh} podemos ver que ocurre cuando activamos la opción -X de ssh.
Esta opción activa el X11 forwarding y, como explica detalladamente la universidad de Pennsylvania en su web\cite{penn}, esta característica activa la posiblidad de 
administrar remotamente una máquina a partir de una interfaz gráfica sin que necesariamente la máquina remota tenga interfaz gráfica.

\vspace{6mm}

He dispuesto en la máquina virtual de Ubuntu Server un servidor de ssh instalando openssh-server. Se puede configurar desde el archivo /etc/ssh/sshd-config, no obstante,
por ahora no cambiaremos nada de la configuración por defecto. Cambiamos la red de la máquina virtual a Host-only para poder acceder al servidor desde el anfitrión.
El puerto 22, el que usa por defecto ssh, ha sido abierto con el comando ufw. Tanto la configuración del puerto como la ip de la máquina pueden verse en 
la Figura \ref{fig:figura7}. 

\begin{figure}[H] 
\centering
\includegraphics[scale=0.3]{Figura7.png}  
\caption{Configuración de la máquina servidor. Realizada el 16 de Abril de 2017.} \label{fig:figura7}
\end{figure}

En caso de tener una máquina anfitriona de Linux, basta con tener instalado el comando ssh y realizar la conexión con ssh nombredeusuario@ipdelamáquina.
En mi caso, la máquina anfitriona tiene instalado Windows 10, luego para poder conectarme con ssh necesitaré el software PuTTY\cite{putty}. Ahora podremos
conectarnos fácilmente introduciendo la ip de la máquina en el programa descargado. 

\begin{figure}[H] 
\centering
\includegraphics[scale=0.3]{Figura44.png}  
\caption{Conexión realizada. Realizada el 16 de Abril de 2017.} \label{fig:figura44}
\end{figure}

Nos conectamos con usuario y contraseña como puede verse en la Figura \ref{fig:figura6}
y ejecutamos el comando gedit. Como Ubuntu Server no tiene instalado gedit nos aparecer el error que sugiere su instalación. No obstante, de ser el anfitrión
un sistema operativo basado en linux, se podría activar el X11 forwarding para ejecutar gedit sin tener que instalarlo en la máquina base.

He investigado si es posible lanzar gedit en Windows con X11 forwarding. En distintas páginas se sugiere la idea de instalar Xming pero no he sido capaz de 
hacer que PuTTY y Xming trabajen juntos.


\begin{figure}[H] 
\centering
\includegraphics[scale=0.3]{Figura6.png}  
\caption{Conexión realizada. Realizada el 16 de Abril de 2017.} \label{fig:figura6}
\end{figure}
%------------------------------------------------




%----------------------------------------------------------------------------------------
%	Cuestión 6
%----------------------------------------------------------------------------------------

\section{Muestre la secuencia de comandos y las modificaciones a los archivos correspondientes para permitir acceder a la consola remota sin introducir la contraseña. Pruebe que funciona.}

En una máquina con sistema operativo basado en Linux, se puede instalar el paquete ssh para llevar a cabo este cometido.
Podemos establecer un login con clave pública/privada \cite{sshkey}, en la máquina que se va a conectar generamos un par de llaves con el 
comando ssh-keygen -b 1024 -t dsa. 1024 hace referencia al tamaño de la clave, cuanto más larga más seguridad habrá.
A continuación mandamos la clave a la máquina que actúa como host con el comando ssh-copy.

\vspace{6mm}

En mi caso, voy a conectarme usando Windows a un servidor de Linux, luego usaré el software PuTTY\cite{putty} para establecer la conexión sin contraseña.
En primer lugar generamos un par de llaves como se observa en la Figura \ref{fig:figura16} y guardamos el archivo generado en algún lugar seguro.

\begin{figure}[H] 
\centering
\includegraphics[scale=0.3]{Figura16.png}  
\caption{Generado de llaves. Realizada el 17 de Abril de 2017.} \label{fig:figura16}
\end{figure}

En el archivo generado aparecerá la clave pública, el siguiente paso es copiar esa clave en el archivo \url{/home/user/.ssh/authorized_keys} con la etiqueta ssh-rsa delante tal y como se muestra en la Figura \ref{fig:figura17}.
Cambiamos los permisos del archivo para mayor seguridad.

\begin{figure}[H] 
\centering
\includegraphics[scale=0.3]{Figura17.png}  
\caption{Guardo la clave en el host. Realizada el 17 de Abril de 2017.} \label{fig:figura17}
\end{figure}
%------------------------------------------------

Ahora indicamos a PuTTY donde está situado el archivo con las claves como se muestra en la Figura \ref{fig:figura18} y añadimos el nombre de usuario permanente.

\begin{figure}[H] 
\centering
\includegraphics[scale=0.3]{Figura18.png}  
\caption{Localización del archivo de claves. Realizada el 17 de Abril de 2017.} \label{fig:figura18}
\end{figure}

Al conectarnos podemos ver que no se nos pide usuario y contraseña, sino que se hace uso de las llaves públicas y privadas.

\begin{figure}[H] 
\centering
\includegraphics[scale=0.3]{Figura19.png}  
\caption{Conexión sin contraseña. Realizada el 17 de Abril de 2017.} \label{fig:figura19}
\end{figure}

%----------------------------------------------------------------------------------------
%	Cuestión 7
%----------------------------------------------------------------------------------------

\section{¿Qué archivo es el que contiene la configuración del servicio ssh?}

Podemos ver en \cite{sshconfig} que el archivo \url{ssh_config} es leido por ssh para establecer la configuración del programa.
Por otra parte, en \cite{sshdconfig}, se nos indica que en el archivo \url{sshd_config} está la configuración del demonio de ssh.

\vspace{6mm}

\subsection{¿Qué parámetro hay que modificar para evitar que el usuario root acceda? Cambie el puerto por defecto y compruebe que puede acceder.}

El parámetro PerminRootLogin modifica el servicio para que el usuario root no pueda acceder, por otro lado, el parámetro Port indica el puerto en el que ssh está escuchando.
Cambiemos entonces el archivo como aparece en la Figura \ref{fig:figura20}

\begin{figure}[H] 
\centering
\includegraphics[scale=0.3]{Figura20.png}  
\caption{Configuración de sshd-config. Realizada el 17 de Abril de 2017.} \label{fig:figura20}
\end{figure}

Para termina de establecer correctamente el puerto 2552 como nuevo puerto de ssh falta abrir los puertos con el comando ufw. Esto permitirá la entrada de remota
al servicio. Cambiamos en PuTTY el puerto de envio de la petición a 2552 como se muestra en la Figura \ref{fig:figura21} y conectamos normalmente.

\begin{figure}[H] 
\centering
\includegraphics[scale=0.3]{Figura21.png}  
\caption{Configuración de PuTTY para nuevo puerto. Realizada el 17 de Abril de 2017.} \label{fig:figura21}
\end{figure}

%------------------------------------------------




%----------------------------------------------------------------------------------------
%	Cuestión 8
%----------------------------------------------------------------------------------------

\section{Indique si es necesario reiniciar el servicio.}

Ssh lee la información y la configuración almacenada en sus archivos cada vez que se ejecuta y se abre el servicio. Por lo tanto, si el servicio está en ejecución y
se alteran algunos componentes de sus archivos de configuración, estos no será efectivos hasta la siguiente ejecución. Además, algunos parametros de dichos archivos
son críticos, es decir, su alteración efectiva conllevaría apagar el servicio ssh e iniciarlo con la nueva característica.

\subsection{¿Cómo se reinicia un servicio en Ubuntu?}

Se recomienda usar systemctl\cite{systemctl} para administrar los servicios. Por ejemplo, para reiniciar el servicio ssh habría que proceder como se muestra en la 
Figura \ref{fig:figura22}.

\begin{figure}[H] 
\centering
\includegraphics[scale=0.3]{Figura22.png}  
\caption{Reiniciar servicio en Ubuntu. Realizada el 17 de Abril de 2017.} \label{fig:figura22}
\end{figure}

\subsection{¿Cómo se reinicia un servicio en CentOS?}

De la misma forma que en Ubuntu, se puede usar el comando systemctl\cite{centosservice} para administrar los servicios. El resultado, similar al de Ubuntu puede verse en la Figura \ref{fig:figura31}.

\begin{figure}[H] 
\centering
\includegraphics[scale=0.3]{Figura31.png}  
\caption{Reiniciar servicio en CentOS. Realizada el 17 de Abril de 2017.} \label{fig:figura31}
\end{figure}



%------------------------------------------------




%----------------------------------------------------------------------------------------
%	Cuestión 9
%----------------------------------------------------------------------------------------

\section{Muestre los comandos que ha utilizado en Ubuntu Server para realizar la instalación de LAMP.}

En Ubuntu es posible instalar los servicios por separado, es decir, instalar Apache2, MySQL y PHP de uno en uno con apt e ir configurándolos. No obstante, existe
una manera más facil, el comando tasksel\cite{tasksel}. Este comando abre una ventana gráfica para instalar algunas aplicaciones, entre esas aplicaciones esta LAMP.
Se puede ver el proceso en las Figuras \ref{fig:figura23}, \ref{fig:figura24} y \ref{fig:figura25}.

\begin{figure}[H] 
\centering
\includegraphics[scale=0.3]{Figura23.png}  
\caption{Selección de aplicaciones a instalar. Realizada el 17 de Abril de 2017.} \label{fig:figura23}
\end{figure}
\begin{figure}[H] 
\centering
\includegraphics[scale=0.3]{Figura24.png}  
\caption{Instalación de LAMP Realizada el 17 de Abril de 2017.} \label{fig:figura24}
\end{figure}
\begin{figure}[H] 
\centering
\includegraphics[scale=0.3]{Figura25.png}  
\caption{Selección de contraseña de administrado para MySQL. Realizada el 17 de Abril de 2017.} \label{fig:figura25}
\end{figure}

En la Figura \ref{fig:figura26} se muestra el servidor web funcionando desde la máquina anfitriona.

\begin{figure}[H] 
\centering
\includegraphics[scale=0.3]{Figura26.png}  
\caption{Servidor web funcionando. Realizada el 17 de Abril de 2017.} \label{fig:figura26}
\end{figure}

Para comprobar que PHP está bien instalado, hacemos un archivo de prueba en /var/www/html/info.php. Comprobamos que en el buscador se efectúa el comando
php.info() al conectarse desde el anfitrión como se muestra en la Figura \ref{fig:figura27}.

\begin{figure}[H] 
\centering
\includegraphics[scale=0.3]{Figura27.png}  
\caption{Comprobación de PHP en Ubuntu. Realizada el 17 de Abril de 2017.} \label{fig:figura27}
\end{figure}

\subsection{Muestre las capturas de pantalla de CentOS correspondientes a la instalación de LAMP.}

Para instalar LAMP en CentOS instalaremos cada componente por separado\cite{lampcentos} con yum. Primero instalaremos apache con \textbf{sudo yum install httpd} y lo iniciaremos\ref{fig:figura28}.

\begin{figure}[H] 
\centering
\includegraphics[scale=0.3]{Figura28.png}  
\caption{Instalación de apache en CentOS. Realizada el 17 de Abril de 2017.} \label{fig:figura28}
\end{figure}

A continuación instalamos la base de datos, en este caso instalaremos MariaDB con el comando \textbf{sudo yum install mariadb-server mariadb} para luego instalar
mysql con el comando \textbf{sudo mysql-secure-installation}. Tras este último comando se efectúa la instalación de mysql y se pedirá que se especifique varias configuraciones, se muestran algunas en la Figura \ref{fig:figura29}.

\begin{figure}[H] 
\centering
\includegraphics[scale=0.3]{Figura29.png}  
\caption{Instalación de mysql en CentOS. Realizada el 17 de Abril de 2017.} \label{fig:figura29}
\end{figure}

Instalamos ahora PHP, de forma similar a los anteriores, con el comando \textbf{sudo yum install php php-mysql} Para probar que php ha sido instalado satisfactoriamente
tenemos el proceso anterior. Creamos un archivo básico de php y lo introducimos en \url{/var/www/html/info.php} y vemos con curl el código de la página. El 
resultado puede verse en \ref{fig:figura30}.

\begin{figure}[H] 
\centering
\includegraphics[scale=0.3]{Figura30.png}  
\caption{Comprobación del funcionamiento de PHP. Realizada el 17 de Abril de 2017.} \label{fig:figura30}
\end{figure}


%----------------------------------------------------------------------------------------
%	Cuestión 10
%----------------------------------------------------------------------------------------

\section{Realice la instalación de IIS y el resto de complementos usando GUI y compruebe que el servicio está funcionando accediendo a la MV a través de la anfitriona.}

Tras instalar el servidor IIS en Windows Server\footnote{No lo muestro con capturas de pantalla porque el proceso está detallado en la práctica y no tiene cambios interesantes} 
y comprobar desde local que el servidor esta activo\footnote{Accediendo en un buscador a la direcion https://localhost},
pasamos a comprobar que está accesible desde otras máquinas. Para ello, desde VirtualBox cambiamos la configuración de red de la máquina y la preparamos para
establecer una conexión de puente o Bridge. Ahora, al iniciar la máquina virtual, la dirección en mi caso es 192.168.56.101 desde la máquina anfitriona. Podemos 
ver en la Figura \ref{fig:figura4} como IIS es accesible desde ambas máquinas.

\begin{figure}[H] 
\centering
\includegraphics[scale=0.3]{Figura4.png}  
\caption{Comprobación del servidor IIS desde anfitrión. Realizada el 15 de Abril de 2017.} \label{fig:figura4}
\end{figure}

%------------------------------------------------




%----------------------------------------------------------------------------------------
%	Cuestión 11
%----------------------------------------------------------------------------------------

\section{Muestre un ejemplo de uso del comando patch.}

Para mostrar un ejemplo del comando patch, he usado la referencia que se aporta en las prácticas\footnote{\url{http://fedoraproject.org/wiki/VMWare}}. 
El script que aparece en la referencia para dar la posibilidad de usar Fedora 20 en una máquina virtual de VMWare contiene el comando patch.
El comando en esta ocasión es usado con los argumentos -i y -p0, para indicar el lugar donde se encuentra la información del parche. Para
más información se puede buscar en el manual en linea con el comando man patch\cite{manpatch}.
Observamos en la Figura \ref{fig:figura5} como el script se ejecuta y comienza a descargarse el parche necesario(comando curl).

\begin{figure}[H] 
\centering
\includegraphics[scale=0.3]{Figura5.png}  
\caption{Ejemplo de uso del comando patch. Realizada el 15 de Abril de 2017.} \label{fig:figura5}
\end{figure}




%------------------------------------------------

%----------------------------------------------------------------------------------------
%	Cuestión 12
%----------------------------------------------------------------------------------------

\section{Realice la instalación de esta aplicación y pruebe a modificar algún parametro de algún servicio. Muestre las capturas de pantalla en ambos procesos.}

Primero descargamos el paquete de instalación que hay en la página web desde la máquina anfitriona. Para poder pasar los archivos hasta la máquina virtualizada, 
usaremos sftp desde la terminal, en mi caso desde PowerShell. Con el comando psftp open miguelangeltorres@192.168.56.101 abro la conexión con la máquina  virtual,
luego, con el comand put archivo, transfiero el paquete hasta esta como se ve en la Figura \ref{fig:figura8}

\begin{figure}[H] 
\centering
\includegraphics[scale=0.6]{Figura8.png}  
\caption{Subir paquete al servidor. Realizada el 16 de Abril de 2017.} \label{fig:figura8}
\end{figure}

Ahora desde la máquina virtual instalamos el paquete con el comando dpkg\cite{dpkg}. Es posible que salgan algunas dependencias no instaladas.
Se resuelve instalando las dependencias primero y luego intentando instalar de nuevo el paquete como se muestra en la Figura \ref{fig:figura9}

\begin{figure}[H] 
\centering
\includegraphics[scale=0.4]{Figura9.png}  
\caption{Instalar el paquete. Realizada el 16 de Abril de 2017.} \label{fig:figura9}
\end{figure}

Una vez instalado, se puede iniciar la interfaz web desde el enlace que especifica por defecto. Se abrirá una página pidiendo usuario y contraseña del administrador y,
a continuación aparecerá la página de webmin como en la Figura \ref{fig:figura10}. Probaremos a alterar los usuarios y los grupos. Para probar, crearemos un usuario 
llamado usuario y un grupo llamado NuevoGrupo al que añadiremos a varios usuarios. El resultado puede verse en la Figura \ref{fig:figura11}.

\begin{figure}[H] 
\centering
\includegraphics[scale=0.3]{Figura10.png}  
\caption{Página inicial de Webmin. Realizada el 16 de Abril de 2017.} \label{fig:figura10}
\end{figure}

\begin{figure}[H] 
\centering
\includegraphics[scale=0.3]{Figura11.png}  
\caption{Nuevo grupo creado en Webmin. Realizada el 16 de Abril de 2017.} \label{fig:figura11}
\end{figure}
%------------------------------------------------


%----------------------------------------------------------------------------------------
%	Cuestión 13
%----------------------------------------------------------------------------------------

\section{Instale phpMyAdmin, inidique cómo lo ha realizado y muestre algunas capturas de pantalla.}

Desde el gesto de paquetes apt se puede descargar phpmyadmin. El comando \textit{sudo apt-get install phpmyadmin} nos conduce a una GUI para descargar y
configurar phpMyAdmin. La instalación viene detallada en la Figura \ref{fig:figura32}.

\begin{figure}[H] 
\centering
\includegraphics[scale=0.3]{Figura32.png}  
\caption{Instalación de phpMyAdmin. Realizada el 16 de Abril de 2017.} \label{fig:figura32}
\end{figure}

Ya tenemos instalado phpMyAdmin. Ahora podemos acceder con el anfitrión en la url \url{http//192.168.56.101/phpmyadmin} y conectarnos con usuario root y contraseña del root de la máquina. Aparece la página inicial de phpMyAdmin como en la Figura \ref{fig:figura33}.

\begin{figure}[H] 
\centering
\includegraphics[scale=0.3]{Figura33.png}  
\caption{Instalación de phpMyAdmin. Realizada el 16 de Abril de 2017.} \label{fig:figura33}
\end{figure}

\vspace{6mm}

\subsection{Configure PHP para poder importar BDs de hasta 25MiB. Indique y muestre como ha hecho el proceso.}

Como se documenta en \cite{postmax},se puede cambiar fácilmente el tamaño máximo de subida de archivos desde el archivo de configuración php.ini situado en el directorio donde se guarda php. La
variable \url{post_max_size} es el tamaño máximo que se permite a un archivo cuando se sube y tiene el valor por defecto 8MB. Se cambia a 25MB con privilegios de 
superusuario y se reinicia el servidor apache2. Ahora la variable ya ha hecho efecto y podemos importar bases de datos de hasta 25MB. El archivo alterado puede
verse en la Figura \ref{fig:figura37}


\begin{figure}[H] 
\centering
\includegraphics[scale=0.3]{Figura37.png}  
\caption{Modificación del tamaño máximo de subida. Realizada el 16 de Abril de 2017.} \label{fig:figura37}
\end{figure}

%------------------------------------------------

%----------------------------------------------------------------------------------------
%	Cuestión 14
%----------------------------------------------------------------------------------------

\section{Visite al menos una de las webs de los software mencionados y pruebe las demos que ofrecen realizando capturas de pantalla y comentando qué está realizando.}

Vistaremos la web de DirectAdmin\footnote{https://directadmin.com/demo.html} y probaremos la Admins demo. Cuando accedemos a la demo, se nos dice la
contraseña y el usuario para poner en el login, al introducirlos, ya podemos ver la página inicial del software\ref{fig:figura34}.

\begin{figure}[H] 
\centering
\includegraphics[scale=0.3]{Figura34.png}  
\caption{Página inicial de DirectAdmin. Realizada el 18 de Abril de 2017.} \label{fig:figura34}
\end{figure}

Vamos a realizar algunas pruebas para ver cómo funciona. Por ejemplo, primero cambiemos la contraseña de la cuenta administrador. Para ello nos vamos al
apartado de contraseñas e introducimos la antigua y la nueva contraseña como podemos ver en la Figura \ref{fig:figura35}.

\begin{figure}[H] 
\centering
\includegraphics[scale=0.3]{Figura35.png}  
\caption{Cambio de contraseña en DirectAdmin. Realizada el 18 de Abril de 2017.} \label{fig:figura35}
\end{figure}

Ahora queremos ver el directorio donde se guardan los archivos del servidor y, por ejemplo, añadir un archivo a este directorio. Accedemos a la pestaña de \textbf{files}.
Vemos ahora el directorio y, en la parte inferior, podemos añadir un archivo. Le damos un nombre y marcamos la casila para que el archivo se rellene con una plantilla
html. Podemos ver en la Figura \ref{fig:figura36} el proceso. Al escribir el archivo y guardarlo, se almacenará en el directorio principal.

\begin{figure}[H] 
\centering
\includegraphics[scale=0.3]{Figura36.png}  
\caption{Creación de archivos en DirectAdmin. Realizada el 18 de Abril de 2017.} \label{fig:figura36}
\end{figure}

%------------------------------------------------

%----------------------------------------------------------------------------------------
%	Cuestión 15
%----------------------------------------------------------------------------------------

\section{Automatización de tareas.}

\subsection{Ejecute los ejemplos de find, grep.}

\begin{figure}[H] 
\centering
\includegraphics[scale=0.3]{Figura38.png}  
\caption{Ejecución de los comandos de ejemplo find y grep. Realizada el 18 de Abril de 2017.} \label{fig:figura38}
\end{figure}

\vspace{6mm}

\subsection{Escriba el script que haga uso de sed para cambiar la configuración de ssh y reiniciar el servicio.}

He escrito un script que cambia la variable PermitEmptyPasswords del archivo \url{/etc/ssh/sshd_config} a si, es decir, configura ssh para que se puedan usar 
contraseñas vacías. El script se puede ver en la Figura \ref{fig:figura40}. Para mostrar un ejemplo de su uso vea la Figura \ref{fig:figura39}.
 
\begin{figure}[H] 
\centering
\includegraphics[scale=0.3]{Figura40.png}  
\caption{Script realizado. Realizada el 18 de Abril de 2017.} \label{fig:figura40}
\end{figure}
\begin{figure}[H] 
\centering
\includegraphics[scale=0.3]{Figura39.png}  
\caption{Ejemplo de funcionamiento del script. Realizada el 18 de Abril de 2017.} \label{fig:figura39}
\end{figure}
\vspace{6mm}

\subsection{Muestre un ejemplo de uso para awk.}

Aunque ya hemos usado awk en el apartado anterior, mostraremos otro ejemplo de uso. Sabemos que el archivo \url{/etc/ssh/sshd_config} contiene una variable 
llamada X11Forwarding, pero no sabemos su valor. En lugar de buscar en el archivo su valor ejecutamos el comando como en la Figura \ref{fig:figura41}

\begin{figure}[H] 
\centering
\includegraphics[scale=0.3]{Figura41.png}  
\caption{Ejemplo del comando awk. Realizada el 18 de Abril de 2017.} \label{fig:figura41}
\end{figure}

%------------------------------------------------

%----------------------------------------------------------------------------------------
%	Cuestión 16
%----------------------------------------------------------------------------------------

\section{Escriba el script para cambiar el acceso a ssh usando PHP o Python.}

En las Figuras \ref{fig:figura42} y \ref{fig:figura43} podemos ver el script apoyado en el ejercicio anterior y la ejecución del mismo. 
\begin{figure}[H] 
\centering
\includegraphics[scale=0.3]{Figura42.png}  
\caption{Script en Python. Realizada el 18 de Abril de 2017.} \label{fig:figura42}
\end{figure}
\begin{figure}[H] 
\centering
\includegraphics[scale=0.3]{Figura43.png}  
\caption{Ejecución del script. Realizada el 18 de Abril de 2017.} \label{fig:figura43}
\end{figure}


%------------------------------------------------

%----------------------------------------------------------------------------------------
%	Cuestión 17
%----------------------------------------------------------------------------------------

\section{Abra una consola de Powershell y pruebe a parar un programa en ejecución, realice capturas de pantalla y comente lo que muestra.}

He realizado la parada de algunos programas en distintas situaciones. En primer lugar, con la cuenta de administrador, he abierto el buscador y, también desde la 
cuenta de administrador he ejecutado un comando para parar el proceso lanzado, Como se puede ver en la Figura \ref{fig:figura1}, el comando ha sido ejecutado con 
éxito, no ha dado ningún mensaje adicional pero observamos que la ventana vinculada al programa se ha cerrado. 

\begin{figure}[H] 
\centering
\includegraphics[scale=0.4]{Figura1.png}  
\caption{Parada de un proceso en ejecución desde administrador. Realizada el 15 de Abril de 2017.} \label{fig:figura1}
\end{figure}

En segundo lugar, he abierto el mismo programa desde la cuenta de administrador. A continuación, he cambiado de usuario y he abierto una cuenta de usuario sin
privilegios. Al intentar cerrar el programa ejecutado desde la cuenta de administrador no salta un error, no se tienen los permisos necesarios para parar el programa.
Puede observarse el resultado en la Figura \ref{fig:figura2}.

\begin{figure}[H] 
\centering
\includegraphics[scale=0.4]{Figura2.png}  
\caption{Parada de un proceso de administrador en ejecución desde cuenta sin privilegios. Realizada el 15 de Abril de 2017.} \label{fig:figura2}
\end{figure}

Por último, he abierto un programa con cuenta de usuario sin privilegios. He intentado pararlo con la propia cuenta de usuario y, como se puede observar en la Figura
\ref{fig:figure3}, el comando ha dado resultados y el programa se ha cerrado.

\begin{figure}[H] 
\centering
\includegraphics[scale=0.4]{Figura3.png}  
\caption{Parada de un proceso de usuario con cuenta de usuario. Realizada el 15 de Abril de 2017.} \label{fig:figura3}
\end{figure}
%------------------------------------------------

\newpage

\bibliography{citando} %archivo citas.bib que contiene las entradas 
\bibliographystyle{plain} % hay varias formas de citar

\end{document}
